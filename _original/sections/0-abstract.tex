%%
%% The abstract is a short summary of the work to be presented in the
%% article.
\begin{abstract}
  % This study examines the individual experiences of extreme PM2.5 exposure among at-risk motorcycle taxi drivers in urban Thailand.
  % Using a mixed-methods approach, we combine seven months of longitudinal sensor data with qualitative interviews to capture both quantitative and qualitative dimensions of air pollution exposure.
  % Our findings reveal that drivers are consistently exposed to PM2.5 levels exceeding WHO guidelines, with limited agency to mitigate these risks due to economic pressures and systemic barriers.
  % We integrate fine-grained sensor data with personal narratives to uncover exposure patterns, coping strategies, and actionable insights.
  % This research highlights the urgent need for interventions to reduce PM2.5 exposure and empower vulnerable workers to prioritize their health while maintaining income stability.


Addressing the gap between gig work autonomy and environmental health agencies, this study investigates extreme PM2.5 exposure among motorcycle rideshare drivers in urban Thailand.
These workers face high risks due to prolonged open-air work with extensive working hours, often rendering standard health advice impractical.
Using a mixed-methods approach combining seven months of longitudinal personal PM2.5 data and qualitative interviews,
we captured detailed exposure patterns and explored lived experiences, health perceptions, coping strategies, and systemic barriers.
We find drivers are consistently exposed to PM2.5 exceeding guidelines,
with limited agency to mitigate these risks due to economic necessity and structural constraints.
Our research underscores the critical need for interventions targeting exposure reduction and worker empowerment,
balancing health protection with income stability for this vulnerable population.
  

  % This study examines the individual experiences of extreme PM2.5 exposure among at-risk motorcycle rideshare drivers in urban Thailand.
  % As open-air vehicle operators spending extensive hours outdoors, these workers face significant health risks from air pollution.
  % Standard public health advice is often impractical for this population, and prior research suggests the promised autonomy of gig work may not translate into the agency needed to mitigate environmental health hazards.
  % This motivates our mixed-methods approach, combining seven months of longitudinal personal air quality sensor data with in-depth qualitative interviews.
  % We aimed to capture detailed, real-time individual exposure patterns and explore the lived experiences, health perceptions, coping strategies, and specific barriers that prevent workers from prioritizing their well-being.
  % Our findings reveal that drivers are consistently exposed to PM2.5 levels exceeding health guidelines, with limited agency to mitigate these risks due to economic pressures and systemic barriers.
  % This research highlights the urgent need for interventions to reduce PM2.5 exposure and empower vulnerable workers to prioritize their health while maintaining income stability.
\end{abstract}