\section{Introduction}

% Motorcycle rideshare is a crucial component of urban transportation in Thailand, where passengers ride open-air vehicles \cite{tieanklin2024rideshare}.
% The drivers, often from lower-income backgrounds, spend extensive hours outdoors in dense urban cores \cite{tieanklin2024rideshare}.
% This mode of work makes them uniquely vulnerable to environmental hazards, particularly prolonged exposure to air pollution.
% High motorcycle ownership (87\% of households in 2014 \cite{Poushter2015motorcyclestat}) and poor air quality rankings\footnote{\url{https://www.iqair.com/us/world-air-quality-ranking}} exacerbate this issue across major cities in the country.

% A primary concern is Particulate Matter 2.5 (PM2.5), fine inhalable particles linked to significant health risks.
% The World Health Organization (WHO) warns that PM2.5 inhalation increases the risk of heart disease, stroke, respiratory illness, and lung cancer, contributing to an estimated 6.7 million premature deaths globally each year \cite{who_ambient_air_pollution}.
% Prolonged exposure is associated with a 6-13\% increase in heart disease mortality per 10 $\mu g/m^3$ of PM2.5.

Motorcycle rideshare drivers in Thailand, often low-income, face prolonged outdoor work in dense, polluted urban areas \cite{tieanklin2024rideshare}.
This vulnerability is amplified by high motorcycle usage (87\% of households in 2014 \cite{Poushter2015motorcyclestat}) and poor air quality \cite{iqr_rank}.
A key hazard is PM2.5, fine particulate matter linked to severe health risks, including respiratory and cardiovascular diseases and millions of premature deaths globally \cite{who_ambient_air_pollution}.
Prolonged exposure is associated with a 6-13\% increase in heart disease mortality per 10 $\mu g/m^3$ of PM2.5.

% Standard public health advice for high PM2.5 events, such as spending more time indoors or avoiding busy roads \cite{cdc_pm25},
% \todo{citation from firn's previous paper; cannot find it anymore.}
% is often impractical for those whose livelihoods depend on being outdoors in these very environments, like motorcycle rideshare drivers.
% % Furthermore, while gig work promises autonomy, this flexibility is often constrained in practice.
% Prior research indicates that financial pressures, platform dynamics, and other factors limit the agency of rideshare drivers,
% preventing them from taking actions—such as reducing work hours during high pollution—to mitigate health risks \cite{tieanklin2024rideshare}.
% This highlights a critical gap between the known risks and the workers' ability to protect themselves.
Standard public health advice for high PM2.5 (e.g., stay indoors and avoid busy roads) \cite{cdc_pm25} is often impractical for these workers.
Financial pressures and platform dynamics limit their agency to prioritize health over income, creating a gap between known risks and self-protection capabilities \cite{tieanklin2024rideshare}.

% In this project, we focus on the individual experiences of extreme PM2.5 exposure among at-risk populations in urban Thailand. Our research employs a robust methodology that combines longitudinal data collection over seven months with qualitative interviews, allowing us to capture both the quantitative and qualitative dimensions of air pollution exposure.

% By deploying personal sensors, we gather detailed, fine-grained data that reflects individual experiences in real-time. These sensors provide multi-modal information about PM2.5 levels, enabling us to monitor variations in exposure across different seasons and environmental conditions. This extended data collection period allows us to observe how PM2.5 levels fluctuate over time and how these fluctuations impact the health and well-being of workers. According to World Health Organization (WHO) guidelines, many individuals in our study are exposed to PM2.5 levels that exceed recommended limits. We will present data later that illustrates the number of days these workers experience exposure above the suggested thresholds, highlighting the urgency of addressing this public health issue.

% A significant outcome of this research project is also community building; we aim to bring together individuals who share similar health concerns related to air pollution, fostering a subcommunity focused on awareness, support, and action.

Understanding the specific, individual experiences of PM2.5 exposure and the barriers preventing workers from prioritizing their health is crucial.
This study aims to bridge this gap by focusing on the lived experiences of at-risk workers (including but not limited to rideshare drivers) in urban Thailand.
We employ a methodology combining longitudinal personal sensing with qualitative interviews to capture both the quantitative patterns and the qualitative nuances of exposure and agency over seven months.
By deploying personal sensors, we gather detailed, real-time data on individual PM2.5 exposure across varying conditions.
This quantitative data is complemented by in-depth qualitative interviews exploring personal narratives, health perceptions, and coping strategies.
We will present data illustrating that workers frequently experience exposure exceeding WHO guidelines \cite{who_aqg_2021}, highlighting the urgency of this issue.
% A potential outcome of this research is also fostering community among participants around shared health concerns focused on awareness, support, and action.

% We make several key contributions through this study:
% \begin{enumerate}
%     % \item \textbf{Sensor Deployment}: Our deployment of personal sensors allows for continuous monitoring of PM2.5 levels over the seven-month study period. This longitudinal approach provides fine-grained insights into individual exposure patterns and environmental conditions, capturing variations that may occur due to seasonal changes or specific events.
%     \item \textbf{Longitudinal Sensor Deployment}: Deployment of personal sensors provides continuous, fine-grained monitoring of individual PM2.5 exposure over seven months, capturing variations due to seasons, location, and individual behavior.

%     % \item \textbf{Data Integration}: We combine findings from user interviews with sensor data to illustrate that these workers consistently experience PM2.5 levels exceeding WHO guidelines. Our analysis encompasses both a big-picture overview of sensor data—grouped by province, time, and season—and an in-depth examination of each individual driver's experience. This integrative approach enables us to identify correlations between qualitative insights and quantitative measurements.
%     \item \textbf{Data Integration}: We integrate quantitative sensor data (analyzed by location, time, and season) with qualitative interview findings to provide a holistic view of workers' exposure patterns and lived experiences relative to health guidelines.
    
%     % \item \textbf{Qualitative Interviews}: We conduct qualitative interviews to gather personal narratives regarding PM2.5 exposure. These interviews provide in-depth insights into how air pollution affects daily life, health perceptions, and coping strategies among workers. By understanding their lived experiences, we can better contextualize the quantitative data collected through sensors.
%     \item \textbf{Qualitative Interviews}: In-depth interviews gather personal narratives on PM2.5 exposure, revealing health perceptions, impacts on daily life, existing coping strategies, and perceived agency (or lack thereof).

%     % \item \textbf{Identifying Challenges and Solutions}: Our findings reveal that individuals currently face significant challenges in managing PM2.5 exposure. Through our qualitative interviews, we uncover specific barriers that hinder effective coping strategies. We explore potential actions that can be taken at both the individual and community levels, such as adjusting driving behaviors to minimize exposure during peak pollution months while maintaining income levels. For instance, drivers could reduce their working hours during high pollution periods and increase them during lower pollution times, thereby balancing their economic needs with health considerations.
%     \item \textbf{Identifying Challenges and Solutions}: We identify specific challenges workers face in managing exposure and explore potential individual and systemic solutions, considering the trade-offs between health and income needed for effective interventions.
% \end{enumerate}
Key contributions include:
\begin{enumerate}
    \item 
Fine-grained, longitudinal individual PM2.5 exposure monitoring.
    \item 
Integration of sensor data with qualitative insights into lived experiences and perceptions relative to health guidelines.
    \item 
Identifying challenges, coping strategies, and agency constraints through worker narratives.
    \item 
Exploring potential solutions considering health-income trade-offs. This research aims to inform interventions enhancing worker agency and well-being.
\end{enumerate}
% (1) Fine-grained, longitudinal individual PM2.5 exposure monitoring. (2) Integration of sensor data with qualitative insights into lived experiences and perceptions relative to health guidelines. (3) Identifying challenges, coping strategies, and agency constraints through worker narratives. (4) Exploring potential solutions considering health-income trade-offs. This research aims to inform interventions enhancing worker agency and well-being.

% % Through these contributions, our research aims not only to enhance understanding of PM2.5 exposure among urban workers in Thailand but also to empower them with knowledge and strategies for making informed decisions about their health and work, and ultimately explore ways to increase their agency to prioritize their health.
% Through these contributions, our research aims to deepen the understanding of PM2.5 exposure for vulnerable urban workers in Thailand, identify barriers to protective action, and explore pathways to increase their agency in prioritizing health alongside economic needs.






















% In this project, we study instead \textit{individual} experiences of extreme pm 2.5 exposure among at-risk populations in urban Thailand.
% Using deployed personal sensors we gather:
% - fine-grained: individual experience, individual sensor, multi-modal sensor information
% - extended period of time: cover multiple seasons with different variations of pm2.5 values
% - according to guidelines from WHO, these workers are exposed to pm2.5 more than the suggested minimum.
%   - (later, show plot on how many (days?) do these workers are exposed over the suggested minimum).
% - consequence of this research project -> community building. We bring together people and create a subcommunity of people with similar interest/concern (health concern).


% we make the following contributions
% \kurtis{include the data collection as a contribution}
% 1) user interview to inquire personal experience with pm2.5 regarding...
% 2) we deploy sensors xxx to measure xxxx, giving us fine-grained insights to xxxx
% 3) combine both results? show that these workers are exposed to pm2.5 more than the minimum guideline (what can we do about the interview result here?)
% - both big-picture analysis of the sensor data on 3 aspects, gorupby province/time/season.
% - in-depth analysis of each individual driver experience
% 4) show that currently these individuals has poor experience dealing with pm2.5. And, what actions can we take for individuals or sub-community.
% - can the drivers change their driving behaviors, by reducing the the driving hours during the peak months and increasing the driving hours during the low pollution months. This way, the amount of income is the same.
% - same for the time during the day.