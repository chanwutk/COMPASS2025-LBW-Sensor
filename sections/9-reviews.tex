\clearpage
\section*{COMPASS Reviews}
\subsection{Review \#45A}
\begin{table}[h]
\begin{tabular}{|l|l|}
    \hline
    \textbf{Overall merit.} &
    2. Weak reject \\
    \hline
    \textbf{Reviewer expertise.} &
    2. Some familiarity \\
    \hline
    \textbf{Paper Relevance to COMPASS} &
    1. Very relevant \\
    \hline
\end{tabular}
\end{table}

\paragraph{Summary}
The study deploys air quality sensor to their understand PM2.5 exposure for low-income motorcycle taxi drivers in Thailand. The core contribution of the work is to improve the level of awareness to the people, most exposed to the air pollution, by providing real-time monitoring. The study focuses on collecting data from helmet mounted air quality sensor for all the riders with an intention to help avoid exposure to air pollution. The study also analyzes the data and conducted interviews with the participants to gauge the impact of the deployment.

\paragraph{Strengths}
\begin{itemize}
\item 
The study performs both quantitative and qualitative analysis of the data regarding the PM2.5 exposure. Results show that although overall exposure doesn’t change, the awareness among the participants helped make changes to their driving pattern to avoid exposure.
\item
The study finds that socio-economic factors are a major hurdle in making changes to driving behaviors.
\end{itemize}

\paragraph{Weakness}
\begin{itemize}
\item
I am unsure of potential impact of this approach. Since pollution is everywhere, changing the route of the rider doesn't affect that much as admitted by some of the participants .
Also, the changing the route depends heavily on the economic factors and incentive from the platform as highlighted by the study. This makes me think whether the nuances which authors claim to be the core contribution are actually feasible in real-world when monitoring air quality.
\mick{In discussion: better policy for the solution?}
\item
The dataset is too small to draw any conclusions drawn here. I think work from [7] does a good job though in terms of number of participants. I think the authors need to be at least somewhat close to that.   
\mick{Our dataset is larger. We have 10 sensors, while they have 6. They have 400 participants for their survey, but we did months-long study.}
\item
The paper was difficult to understand and unnecessarily long which made it difficult to follow. The authors need to do better framing of the problem. The writing needs to be improved significantly.
\mick{We might want to cut the parts on emotional aspects of the participants (and maybe community building). Then, we can have a clearer objective of how to help drivers make healthier decisions without income tradeoffs.}
\end{itemize}


\clearpage
\subsection{Review \#45B}
\begin{table}[h]
\begin{tabular}{|l|l|}
    \hline
    \textbf{Overall merit.} &
1. Reject \\
    \hline
    \textbf{Reviewer expertise.} &
4. Expert \\
    \hline
    \textbf{Paper Relevance to COMPASS} &
2. Relevant \\
    \hline
\end{tabular}
\end{table}

This paper presents a study about motorcycle taxi drivers in Thailand using helmet mounted sensors to monitor their exposure to pm2.5. Using the sensor values and findings from qualitative studies that focus on user experience with the sensors and the data platform, the authors present results on times and locations with highest pollution plus the social implications of having access to air quality data.

\paragraph{Strengths}
\begin{itemize}
\item
the findings around the social uses of the data, including across cities and changing driver patterns is interesting and shows the potential of data access
\item
the interview coding section is nicely detailed
\item
the stated contributions are well laid out
\end{itemize}

\paragraph{Areas of opportunity}
\begin{itemize}
\item
the abstract is missing from the paper
\mick{fixed}
\item
the introduction is too short and does not properly introduce the problem (air pollution) and motivation behind this work. It should read like a summary of the paper to help guide the reader but is missing a lot of information
\item
the related works feel more like a critique of a few prior studies rather than an engagement with literature. I’d like to know 1) why the chosen topics are relevant to this work and 2) how this work fills a gap that emerges from prior work. I also don’t think the comparison to a paper about global poverty versus air pollution is the right one to make here \todo{Firn is at this}
\item
it also just feels like there needs to be more examination into prior work overall. The authors should compare to other mobile sensor studies, and be sure to indicate works they are referencing, i.e.  in section 5.1 
\item
I don’t understand the proposal of discounts during high pollution times mentioned in section 5.4. It seems this would just expose drivers to more pollution 
\item
I’m not sure I understand the point of figure 4. Its also confusing that it is referenced after figure 5 in the text so I don’t know if that is a textual error \todo{Mick is at this}
\item 
table 1 is not really demographic information, more employment information
\item
the authors state “ This results indicate that drivers choose their service location in a way that avoid locations with high PM 2.5 levels.” In response to qs2, but this seems at odds with the prior quote that said drivers just go where they need to
\end{itemize}

\paragraph{Minor}
\begin{itemize}
\item
typos throughout
\item
one sentence that says “exposure…??”
\item
starting a related works section with “this work” but that refers to a prior work that hadn’t been introduced yet
\item
unanonymized institution for IRB
\end{itemize}

Overall, I like the goals of this paper but feels that it needs a lot more work to be ready for publication. I hope the authors will take time to revise and make this feel like a full paper that lives up to its stated contributions then submit again.


\clearpage
\subsection{Review \#45C}
\begin{table}[h]
\begin{tabular}{|l|l|}
    \hline
    \textbf{Overall merit.} &
3. Weak accept \\
    \hline
    \textbf{Reviewer expertise.} &
2. Some familiarity \\
    \hline
    \textbf{Paper Relevance to COMPASS} &
3. Somewhat relevant \\
    \hline
\end{tabular}
\end{table}

This paper discusses a study run over seven months with motorcycle rideshare drivers in Thailand using a helmet-mounted air quality sensor. The paper discusses the findings from the study in a mixed methods approach, showing quantitative data of air quality alongside qualitative interview data.

The study is relevant to COMPASS, I believe, but perhaps doesn't quite address an HCI audience as we would hope.

I think this study is rigorous and very interesting. I loved seeing the people make sense of data and make decisions (or not) using this new information. I think the length of time and quantity of people surveyed, as well as the compensation for the study, are all commendable. I also found the detailed insights in the findings section fascinating and rich. As a qualitative researcher, I was less intrigued by the quant findings (although they do help tell a story of the rhythms of the seasons and days) – and was really invested in section 4.3 which was a wonderful ethnographic look at different ways of encountering this air quality data.

I feel a bit lost as to who the audience is and this is evident in the shaping and framing of the paper and the discussion and take-aways. The shaping and framing of the paper, which would often be structured in the related works, show a clear lack of relationship to HCI research – at least from the ACM cannon. This results in a lack of relevant findings and framing for the HCI community. For example, I can imagine the paper being written in a different way about how access to real-time air quality data shifts user behavior as an HCI researcher. However, here the research question and research outcome seem to be more about describing a phenomenon and less about the impact of technologies on drivers. The authors claim the two contributions are about gathering health information about PM2.5 exposure and building community. There is, therefore, a lack of initial framing that draws attention to the impact of the technological device. However, the impact is quickly made obvious in the findings as riders shift their behavior by wearing masks, check for air quality data to avoid hotspots, or simply continue to drive into pollution in spite of the new information. I was left thinking “how can I use this?” as an HCI researcher . . . but I wonder if the authors come from a slightly different discipline as this research is (frankly) a bit more long-term and robust than much of HCI’s research. And I was wondering if that is ok in COMPASS or if we need to hold authors to the ACM/HCI cannon and audience a bit more.

I would love to see a positionality statement just because I am curious about the researcher’s relationship to these people and how it formed. I would love to hear a bit more about the sensor earlier on – what sensors were used, how were they attached to the helmet, how did they send data to the app? How did participants access that data?

Over all, I thought it was great, yet a little ‘off’ from a traditional HCI paper. I wonder if that is ok in a venue like COMPASS? Looking forward to discussing.


\clearpage
\subsection{Meta Review -- 2. Weak reject}
The reviewers appreciated seeing this work as the topic is related to COMPASS, yet there were several issues that should be addressed to make the paper publish-ready. The reviewers found a lot of the text confusing or missing (i.e., abstract), and the authors should consider their audience and the writing style and format to reach that audience. As Reviewers B and C noted, there are not enough prior works and the related works section needs to be revisited. Although the qualitative findings were interesting, there were not a lot of participants (RA) and the findings were confusing when presented with the quantitative findings (RC).

Given these critiques, the reviewers do not recommend this paper for acceptance this cycle but hope the authors will rewrite and resubmit, focusing on the qualitative findings to better tell their story.