\subsection{Communal Engagement -- From Technical Support to a Community Space}
\label{sec:result-communal-engagement}
% Group Interactions and Air Quality Discourse

In this subsection, we explore how motorcycle taxi drivers in Bangkok and Chiang Mai engaged in collective sensemaking through their group chat interactions, leveraging air quality data from their sensors. 

\subsubsection{Sense of Community and Social Support}
Initially created as a technical support channel, the group chat quickly evolved into a vital community space where drivers from both provinces, Bangkok and Chiang Mai, shared their day-to-day driving experiences, unexpected circumstances, traffic congestion and most importantly, warned each other about worsening air pollution conditions in their respective areas.  

During peak pollution days, the chat became particularly active, with drivers providing real-time updates. On December 11, 2024, one of the first major pollution waves of the season in Bangkok, driver BKK-5 began their day by checking the air quality web portal that we provided and alerting everyone in the groupchat,

\begin{quoteb}
    ``Today the air pollution is really thick, it's very red.'' (BKK-5)
\end{quoteb}

\mick{can remove this paragraph because does not make the point for the communal engagement.}
BKK-5 Driver also reported during the exit interview that he often checked the map visualization showing the air pollution level every day and constantly checked for updates throughout the day, especially during the peak period in January and February.

Beyond leveraging the data that they have collectively collected, the group chat functioned as a space for social support, where drivers expressed concerns about health risks and reminded other participants, sometimes the research team included, to prepare protective measures mitigating exposure. 
This illustrates how the integration of sensor data into social communication channels can enhance risk awareness and foster a sense of community among participants.


\subsubsection{Shared Situational Awareness and Collaborative Interpretation}
Beyond warnings, drivers also used the visualization tools to compare pollution levels between the two cities, Bangkok and Chiang Mai, fostering a shared understanding of exposure risks. 
% Some expressed concerns about their peers’ conditions, reinforcing a sense of solidarity despite being geographically apart.
Our participants reported developing a new habit of frequently checking both their own and other participants' air quality measurements using the map visualization web application, particularly during downtime while waiting for the next ride request.
For instance, BKK-4 was the first to notice the onset of one of the worst air pollution waves in Chiang Mai of 2024, where sensor data showed PM2.5 levels exceeding 200 $\mu g/m^3$ during the day. BKK-4 remarked,

\begin{quoteb}
``(Air pollution in) Chiang Mai has been getting a lot worse a lot lately.'' (BKK-4)
\end{quoteb}

Drivers often use the visualization web application not only to navigate their immediate environment but also to stay informed about broader regional trends.

This emergent community dynamic highlights that drivers are not indifferent to air pollution risks; rather, as discussed earlier, they perceive it as an unavoidable part of their job. Given that motorcycle taxi driving is their primary source of income, individual avoidance strategies remain limited, further underscoring the need for broader structural interventions. 

\subsubsection{Leverage The Tool for Income}
Drivers frequently utilize the air quality map to identify ``red'' zones, which indicate high pollution levels, and infer traffic congestion in those areas. This capability allows them to strategize their routes to maximize income by avoiding congested areas while simultaneously minimizing exposure to poor air quality. 
The interviews with drivers have revealed the dual benefits of using the map visualization web application - optimizing income by increasing the number of pickups while also subsequently  reducing the health risks from the air pollutions as they would just avoid the area.
For instance, one of the most-earning Bangkok drivers noted during the exit interview,

\begin{quoteb}
``It will actually be even more helpful to actually show the traffic real-time, so I can avoid the area altogether...  I can make more rounds of pickups and also reduce the air pollution intake too.'' (BKK-5)
\end{quoteb}


The drivers’ ability to interpret air quality data in conjunction with traffic conditions reflects a strategic approach to their work, where financial incentives take precedence over health considerations. 
This aligns with the previous research for this particular group indicating that gig workers often prioritize immediate income due to precarious employment conditions, leading them to overlook long-term health implications \cite{tieanklin2024rideshare, zhang2022algorithmic}. 
This underscores the need for targeted interventions that address both economic and health concerns for gig workers in urban environments.