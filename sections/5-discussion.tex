\section{Discussion}

% This study, by nature of being both sensor and interview based, provides a uniquely multi-modal mixed methods view in the experience of being a motorcycle driver in Thailand.
% While not explicitly a design study, a key goal of ours was to better understand this environment in order to effect future change on drivers; potentially finding levers or systems that can be used to improve the livelihood of these and other at-risk populations.
% Towards this end, we have a number of discussion points, from surprising use reactions to data to potential policy and infrastructure tools for improving driver health.
This study combines sensor data and interviews to offer a multi-modal perspective on Thai motorcycle rideshare drivers' experiences.
% While not primarily a design study, a
A key goal was understanding this environment to identify interventions improving driver health and livelihoods.
We discuss driver interest in real-time data, policy implications, infrastructure improvements, and platform-level incentives.

\todo{what specific interventions-technological, regulatory, or organizational-do drivers themselves propose or prefer?}

\todo{How feasible are sensor-based early warning systems or adaptive work scheduling in practice?}

\todo{Are there gendered or age-related differences in exposure or agency?}

% \subsection{User Interest in Real-Time Data}
\para{User Interest in Real-Time Data}
% Given that prior work showed little potential for technology-led behavior change from drivers (as they were mostly revenue-sensitive), we were surprised to see that low-cost air quality sensors were an appreciated data source that empowered drivers to make more informed decisions, given drivers expressed more awareness about the issue as they are to see their air quality records.
% Drivers expressed during interviews that it would be helpful in deciding routes if the real-time measurement could show up on the dashboard of the motorcycle, integrate them on real-time air pollution data into rideshare platforms' driver apps, displayed on an interactive map, or even integrated directly into their motorcycles themselves.
% The integration of predictive air quality algorithms into drivers' workflows could provide drivers with proactive warnings about impending high-pollution events, allowing them to adjust their schedules in advance.
% Designing such interventions would be complicated, as many obvious designs (e.g., phone-based interfaces) would likely be distracting and endanger drivers, but it seems like a fruitful area for future work. 
Despite prior work~\cite{tieanklin2024rideshare} suggesting drivers prioritize revenue over technology-led behavioral change, participants valued real-time air quality data, feeling it empowered more informed decisions and increased awareness.
Drivers suggested integrating this data via motorcycle dashboards, ride-share apps, interactive maps, or directly into motorcycles.
Predictive algorithms could offer proactive warnings.
While designing non-distracting interfaces is challenging, this area warrants future investigation.

% \subsection{Government Policy and 
% \todo{shorten}
% Work-from-Home Campaigns}
% The government of Thailand has enacted a number of policies to support citizens in reducing the health impacts of air pollution.
% One example is government-initiated work-from-home (WFH) policies.
% For instance, during Bangkok’s WFH campaign in early 2024, approximately 200 companies and 100,000 employees adopted WFH arrangements, leading to an 8\% reduction in traffic congestion \cite{Wipatayotin_2025}. 
% This decrease in traffic contributed to lower PM2.5 levels and benefited the general population, including motorcycle taxi drivers. 

% Though it is important to recognize that an overall reduction in air pollution will not fundamentally resolve the issue of frontline workers having to work in a poisonous environment to survive, but having less congestion on the roads could still help these workers, motorcycle taxi drivers in our case, reduce their health risk. 
% Such initiatives underscore that individual actions can only be part of the answer; policy changes are needed to effect collective change for citizens as a whole, which indirectly reduces at-risk citizen exposure.
% Additionally, public-private partnerships can facilitate the development and deployment of new technologies for drivers at scale.
% \subsection{Government Policy and Collective Action} % Shortened title
\para{Policy Interventions and Financial Support}
Government interventions like targeted financial support (e.g., subsidies for protective equipment, healthcare benefits, compensation during severe pollution) could improve driver outcomes.
Stricter emission controls and encouraging electric motorcycles offer long-term improvements.
Ultimately, reducing driver exposure requires a multi-stakeholder approach involving government, platforms, passengers, and drivers, integrating technology and policy to balance stakeholder needs (platform profitability, driver health/income, user convenience).

\para{Local and Collective Action}
The differing primary PM2.5 sources were identified; consistent traffic emissions in Bangkok versus seasonal agricultural burning dominating in Chiang Mai. 
This fundamentally shapes the air pollution challenge across regions and underscores that effective mitigation strategies must be location-specific. 
For instance, government policies like Work-From-Home (WFH) campaigns (e.g., Bangkok's 2024 initiative contributed to an 8\% traffic reduction  \cite{Wipatayotin_2025}) directly target the traffic congestion identified as Bangkok's primary issue, particularly during peak rush hours indicated by daily cycles (\autoref{fig:hourly-work-aqi}). 
Yet, these policies offer limited impact where sources differ (e.g., Chiangmai's burning season) and fail to resolve direct exposure for frontline workers like motorcycle taxi drivers who remain operational amidst pollution sources.
Therefore, effective solutions need integrated approaches by complementing national policies with targeted, local actions against dominant sources (traffic/burning) plus direct support (e.g., PPE, technology) for those highly exposed, potentially scaled through tailored public-private partnerships.

% % \subsection{Driver Downtime and Planning}
% % The data makes it clear that a significant amount of exposure happens while drivers are between tasks, waiting on corners, or at restaurants for their next job.
% % Because of this, infrastructure improvements tailored for motorcycle taxi drivers could also help mitigate exposure.
% % For example, creating designated waiting zones with air filtration systems in high-traffic areas like malls, bus stations, or markets would allow drivers to rest in cleaner air while waiting for passengers.
% % Restaurants could even be encouraged to be ``hubs'' for drivers to wait at, potentially offering clear air areas during the downtime.
% % This type of community infrastructure, which may have been difficult to develop in a world where the drivers were in direct competition with each other for jobs, could potentially be easier in a world where much of the decision-making is centralized.
% % Drivers could even be encouraged or paid to spend time in these clean zones to balance greater exposure. 

% % We note that these lines of thought seem somewhat bleak to the researchers, in that these drivers should probably just stop driving if the health effects of driving are so bad, but most jobs have breaks and often those breaks are taken to reduce the chance of injury to workers. 

% % %On a more personal level, portable air-purifying helmets, though currently costly, could become viable with government subsidies or partnerships with rideshare platforms.
% % \subsection{Driver Downtime and Infrastructure} % Shortened title
% \para{Driver Downtime and Infrastructure}
% \mick{revisit: have not mentioned anything about drivers' downtime in the paper.}
% Data reveals that significant pollution exposure occurs while drivers wait between tasks.
% Infrastructure improvements, such as designated waiting zones with air filtration systems in high-traffic areas or encouraging restaurants to serve as clean-air ``hubs,'' could mitigate this.
% Centralized platforms might facilitate developing such infrastructure, perhaps even incentivizing drivers to use these zones.
% While drivers facing severe health effects might ideally stop driving, improving brake conditions aligns with standard occupational safety practices aimed at reducing harm.

% \subsection{Passenger Awareness and Platform-Level Incentives}
% It is important to understand that these app platforms are two-sided markets, with the companies managing both drivers and passengers.
% Because of this, the passengers, drivers, and the platform itself all have shared and competing incentives across the platform.
% For example, during high pollution events potential passengers may wish to stay at home and get food delivered, which requires drivers to be working in the highly polluted environment.
% As such, educating passengers about the health risks of air pollution and humanizing drivers could indirectly benefit drivers.
% For instance, platforms could offer discounts for off-peak hours during high-pollution periods, reducing demand when exposure risks are highest.
% This approach has the potential to align passenger behavior with drivers' health considerations, but at considerable risk to the platforms who may be leery to disincentivize immediate participation on the platforms.

% Alternatively, other actors (e.g., government or NGO) could explore education initiatives among the general populace suggesting avoiding using these platforms during peak pollution hours. We are pursuing this line of thought with a local Thai air pollution NGO.

% %\reread{Platforms can also reward drivers who adopt protective measures, such as wearing masks or using air-purifying helmets, through financial incentives or recognition programs. Such measures would encourage widespread adoption of safety practices.}
% \subsection{Passenger Awareness and Platform-Level Incentives}
\para{Passenger Awareness and Platform-Level Incentives}
Ride-share platforms involve competing passenger and driver incentives.
For example, passenger demand for food deliveries during high pollution increases driver exposure.
Educating passengers about these risks could help.
Platforms could offer off-peak discounts during high-pollution events, aligning incentives, though potentially reducing platform usage.
Alternatively, external campaigns could discourage peak-pollution usage.
% Platforms could also directly incentivize drivers using protective measures (e.g., masks, filtered helmets) through rewards or recognition. % Integrated commented text

% \subsection{Policy Interventions and Financial Support}
% Finally, given how driver's participation with the system is individualistic in nature, government intervention through targeted financial support for vulnerable workers could significantly improve their outcomes.
% Subsidies for protective equipment, healthcare benefits, or direct compensation during extreme pollution periods could alleviate the financial burden on drivers and allow them to make healthier choices.
% Additionally, implementing stricter emission control policies for vehicles and encouraging a transition to electric motorcycles could substantially improve roadside air quality in the long term.

% This discussion emphasizes that reducing the exposure of motorcycle taxi drivers to air pollution requires a multi-stakeholder approach involving governments, platform providers, passengers, and the drivers themselves. 
% By combining innovative technology, infrastructure improvements, and policy interventions, it is possible to balance all of the stakeholder needs; including profitability of the platform, profit and health for the drivers, and convenience for the users. 
% \subsection{Policy Interventions and Financial Support}


% \subsection{Study Limitations}
% \todo{shorten}

% One of the limitation of this study is related to the power source of the air quality sensor. 
% The power source is a critical component of our design, as it must provide sufficient capacity while remaining compact and lightweight to ensure portability for the riders. 
% Our system is designed to operate for approximately 8-10 hours to accommodate the typical working hours of each drivers. 
% To achieve this, we selected a power bank for its convenience, which allows for easy battery swaps or recharge during the day if needed.

% Our air helmet consumes 3.5 watt-hours and is powered by a 5V supply from the battery. After evaluating the power requirements, we opted for a 10,000 mAh battery with a discharge capacity of 36 watt-hours, which is sufficient to support about 8 hours of operation per day. 
% However, given that our drivers reportedly work anywhere from 6 to 17 hours a day, we encouraged our participants to recharge the power bank on their breaks. 
% However in some cases where they were unable to do so, this may result in missing data during late-night hours.
% Despite this limitation, we believe that the data collected is sufficient for this study to understanding how individuals navigate air pollution and the levels of exposure they experience on the streets throughout their days. 
% Nonetheless, we will explore ways to provide a more direct power source for the sensors, such as mounting them directly to the motorcycle.
% This would enhance data collection capabilities and allow for continuous monitoring throughout longer working hours.

% While this study provides a uniquely deep, longitudinal view in to the experience of being a ride-share driver in a major urban area of Thailand, it is not a randomized sample that is able to make generalizable health conclusions as is common in most air quality exposure studies.
% This was intentional from the authors; it's clear that air pollution is a major harm but it was less clear to us how those impacts manifest in the daily lives of drivers.
% A different study would be needed to show statistically valid health impacts on drivers. 

% Additionally, air pollution comprises a much larger span of potentially injurious compounds and our sensor narrowly focused on just a few. We expect that driving for ten hours a day is probably worse for you than our study shows.  
% \subsection{Study Limitations}
\para{Study Limitations}
Limitations include the sensor's battery power (8-10 hours), potentially causing data gaps during longer shifts despite recharge encouragement; direct motorcycle power is a future consideration.
Our non-random sample intentionally focused on uniquely deep, longitudinal lived experiences rather than generating statistically generalizable health conclusions typical of epidemiological studies.
Furthermore, sensors measured only a subset of pollutants, likely underestimating total exposure risks.