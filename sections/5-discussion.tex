\section{Discussion}

This study combines sensor data and interviews to offer a multi-modal perspective on Thai motorcycle rideshare drivers' experiences.
A key goal was understanding this environment to identify interventions improving driver health and livelihoods.
We discuss driver interest in real-time data, policy implications, infrastructure improvements, and platform-level incentives.

\todo{what specific interventions-technological, regulatory, or organizational-do drivers themselves propose or prefer?}
\mick{I don't think they do?}
\mick{probably a better pay, but that does not solve the problem about the air pollution.}
\mick{they also said that they will drive no matter what}

\todo{How feasible are sensor-based early warning systems or adaptive work scheduling in practice?}
\mick{can say that participants are not bothered by the sensor?}
\mick{for charging, if we will be implementing this program, can we provide a battery charger to the driver? Maybe we can have a station to pick up / drop off a charger?}
\mick{may be we can add a paragraph here in the discussion section}

\todo{Are there gendered or age-related differences in exposure or agency?}
\mick{I think we have this data.}
\mick{Firn also brought this when we compared 2 types of drivers.}
\mick{If we don't want to make a point about gender / age, we can ignore this comment.}

\para{User Interest in Real-Time Data}
Despite prior work~\cite{tieanklin2024rideshare} suggesting drivers prioritize revenue over technology-led behavioral change, participants valued real-time air quality data, feeling it empowered more informed decisions and increased awareness.
Drivers suggested integrating this data via motorcycle dashboards, ride-share apps, interactive maps, or directly into motorcycles.
Predictive algorithms could offer proactive warnings.
While designing non-distracting interfaces is challenging, this area warrants future investigation.


\para{Policy Interventions and Financial Support}
Government interventions like targeted financial support (e.g., subsidies for protective equipment, healthcare benefits, compensation during severe pollution) could improve driver outcomes.
Stricter emission controls and encouraging electric motorcycles offer long-term improvements.
Ultimately, reducing driver exposure requires a multi-stakeholder approach involving government, platforms, passengers, and drivers, integrating technology and policy to balance stakeholder needs (platform profitability, driver health/income, user convenience).

\para{Local and Collective Action}
The differing primary PM2.5 sources were identified; consistent traffic emissions in Bangkok versus seasonal agricultural burning dominating in Chiang Mai. 
This fundamentally shapes the air pollution challenge across regions and underscores that effective mitigation strategies must be location-specific. 
For instance, government policies like Work-From-Home (WFH) campaigns (e.g., Bangkok's 2024 initiative contributed to an 8\% traffic reduction  \cite{Wipatayotin_2025}) directly target the traffic congestion identified as Bangkok's primary issue, particularly during peak rush hours indicated by daily cycles (\autoref{fig:hourly-work-aqi}). 
Yet, these policies offer limited impact where sources differ (e.g., Chiangmai's burning season) and fail to resolve direct exposure for frontline workers like motorcycle taxi drivers who remain operational amidst pollution sources.
Therefore, effective solutions need integrated approaches by complementing national policies with targeted, local actions against dominant sources (traffic/burning) plus direct support (e.g., PPE, technology) for those highly exposed, potentially scaled through tailored public-private partnerships.






\para{Passenger Awareness and Platform-Level Incentives}
Ride-share platforms involve competing passenger and driver incentives.
For example, passenger demand for food deliveries during high pollution increases driver exposure.
Educating passengers about these risks could help.
Platforms could offer off-peak discounts during high-pollution events, aligning incentives, though potentially reducing platform usage.
Alternatively, external campaigns could discourage peak-pollution usage.








\para{Study Limitations}
Limitations include the sensor's battery power (8-10 hours), potentially causing data gaps during longer shifts despite recharge encouragement; direct motorcycle power is a future consideration.
Our non-random sample intentionally focused on uniquely deep, longitudinal lived experiences rather than generating statistically generalizable health conclusions typical of epidemiological studies.
Furthermore, sensors measured only a subset of pollutants, likely underestimating total exposure risks.
