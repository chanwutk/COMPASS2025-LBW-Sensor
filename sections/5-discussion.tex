\section{Discussion}

This study combines sensor data and interviews to offer a multi-modal perspective on Thai motorcycle rideshare drivers' experiences.
A key goal was understanding this environment to identify interventions improving driver health and livelihoods.
We discuss driver interest in real-time data, policy implications, infrastructure improvements, and platform-level incentives.




\para{User Interest in Real-Time Data}
Despite prior work~\cite{tieanklin2024rideshare} suggesting drivers prioritize revenue over technology-led behavioral change, participants valued real-time air quality data, feeling it empowered more informed decisions and increased awareness.
Some drivers used this data to prepare for better protection, such as wearing masks on high-pollution days.
However, many also stated they would continue working regardless of conditions, as income needs outweighed health concerns. 
Drivers suggested integrating air quality data via motorcycle dashboards, ride-share apps, interactive maps, or directly into motorcycles.
Predictive algorithms could offer proactive warnings.
While designing non-distracting interfaces is challenging, this area warrants future investigation.



\para{Policy Interventions and Financial Support}
Government interventions like targeted financial support (e.g., subsidies for protective equipment, healthcare benefits, compensation during severe pollution) could improve driver outcomes.
Stricter emission controls and encouraging electric motorcycles offer long-term improvements.
Drivers themselves did not propose specific policy solutions beyond requesting better compensation or basic protections,
reflecting their limited agency in shaping broader environmental conditions.
Ultimately, reducing driver exposure requires a multi-stakeholder approach involving government, platforms, passengers, and drivers, integrating technology and policy to balance stakeholder needs (platform profitability, driver health/income, user convenience).

\para{Local and Collective Action}
The differing primary PM2.5 sources were identified; consistent traffic emissions in Bangkok versus seasonal agricultural burning dominating in Chiang Mai. 
This fundamentally shapes the air pollution challenge across regions and underscores that effective mitigation strategies must be location-specific. 
For instance, government policies like Work-From-Home (WFH) campaigns (e.g., Bangkok's 2024 initiative contributed to an 8\% traffic reduction  \cite{Wipatayotin_2025}) directly target the traffic congestion identified as Bangkok's primary issue, particularly during peak rush hours indicated by daily cycles (\autoref{fig:hourly-work-aqi}). 
Yet, these policies offer limited impact where sources differ (e.g., Chiangmai's burning season) and fail to resolve direct exposure for frontline workers like motorcycle taxi drivers who remain operational amidst pollution sources.
Therefore, effective solutions need integrated approaches by complementing national policies with targeted, local actions against dominant sources (traffic/burning) plus direct support (e.g., PPE, technology) for those highly exposed, potentially scaled through tailored public-private partnerships and advocacy from local nonprofit organizations.








\para{Passenger Awareness and Platform-Level Incentives}
Ride-share platforms involve competing passenger and driver incentives.
For example, passenger demand for food deliveries during high pollution increases driver exposure.
Educating passengers about these risks could help.
Platforms could offer off-peak discounts during high-pollution events, aligning incentives, though potentially reducing platform usage.
Alternatively, external campaigns could discourage peak-pollution usage.








\para{Study Limitations}
Limitations include the sensor’s battery life (8–10 hours), which may lead to data gaps during longer shifts.
Although drivers did not find the sensors intrusive, long-term deployment would benefit from logistical support such as battery swap stations for the sensors' battery power.
Our non-random sample intentionally focused on uniquely deep, longitudinal lived experiences rather than generating statistically generalizable health conclusions typical of epidemiological studies.
Furthermore, sensors measured only a subset of pollutants, likely underestimating total exposure risks.
