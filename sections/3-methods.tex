\section{Methodology}

This study combined real-time sensor deployment for hyperlocal air pollution monitoring with qualitative interviews capturing participant experiences and perceptions.
University of Washington Institutional Review Board (IRB) approval was obtained.

\subsection{Participants Recruitment}

Ten adult motorcycle rideshare drivers (five each from Bangkok/Chiang Mai) were recruited to capture diverse driving behaviors and exposure patterns.
Inclusion criteria were regular passenger transport and driving $\ge$5 hours/day; passengers-dropoff drivers were preferred.
Participant income details are in \autoref{tab:demo}.

\begin{table}[ht]
    \centering
    \caption[Participant Income]{Participant income sources (Rideshare: app-based; Legacy: station-based) and self-estimated daily average (THB).
    DriverID prefix indicates location (BKK: Bangkok, CMI: Chiang Mai).}
    \begin{tabular}{|c|l|c|}
        \hline
        \textbf{DriverID} & \textbf{Source of Income} & \textbf{Avg. Income/} \\
         &  & \textbf{Day (THB)} \\
        \hline
        BKK1  & Rideshare, Legacy & 2250 \\
        BKK2  & Legacy & 800 \\
        BKK3  & Rideshare, Legacy & 1300 \\
        BKK4  & Rideshare, Legacy & 1350 \\
        BKK5  & Rideshare & 1800 \\
        CMI1  & Rideshare & 700 \\
        CMI2  & Rideshare & 700 \\
        CMI3  & Rideshare & 600 \\
        CMI4  & Small Business Owner, Rideshare & 500 \\
        CMI5  & Rideshare & 900 \\
        \hline
    \end{tabular}
    \label{tab:demo}
\end{table}

In Bangkok, participants were initially recruited through a widely used motorcycle ride-hailing application. 
Drivers were approached during rides, with an author (a native Thai speaker) explaining the study objectives and eligibility criteria. 
Due to last-minute participant dropouts, the research team recruited an additional motorcycle rideshare driver from the legacy rideshare system, where drivers wait for passengers at designated stations. 

In Chiang Mai, where the rideshare driver population is smaller, recruitment was conducted through an unofficial rideshare driver association. 
While recruitment was facilitated by the organization, all participants were required/preferred to meet our study criteria stated above.

\paragraph{On-boarding Process}

Interested drivers received study details (purpose, procedures, data handling) and provided informed consent.
Participants were required to sign an informed consent form before participating in the study.
Participants were trained on accessing/interpreting their real-time data via a map visualization~\cite{mapvis}.
We clarified non-affiliation with rideshare platforms and informed participants of their right to withdraw or request data deletion anytime without penalty.





\paragraph{Compensation}
Participants were compensated 4,000 Baht per month for their participation, with payments distributed twice monthly during check-ins by transferring into their bank account.

\subsection{Real-Time Sensor Deployment and Calibration}

To collect real-time air pollution data, we deployed 10 portable low-cost air quality sensors developed by our research team~\cite{pA2025sensor}.






\paragraph{Sensor Deployment and Support}
Sensors were mounted on helmets of 10 motorcycle rideshare drivers (5 Chiang Mai, 5 Bangkok) near the breathing zone, powered by 2 portable batteries during their 6-17 hour workdays.
Drivers manually operated sensors and used a LINE group (a popular messaging app in Thailand) for communication and troubleshooting assistance, prompted by automated malfunction notifications.
They could view real-time collective data via a web application featuring a map visualization tool~\cite{mapvis} displaying color-coded pollution levels; training was provided.
Deployment spanned November 2023 to mid-May 2024, covering Thailand's peak pollution season (Jan-Apr) and diverse seasonal conditions to capture varied PM2.5 exposure.

\paragraph{Bi-Weekly Check-Ins}
Biweekly check-ins facilitated participant feedback, shared data summaries (local street-level pollution), equipment checks, and re-calibration.




\subsection{Qualitative Interview}

Qualitative exit interviews with all participants explored their experiences, routines, and perceptions of air pollution (pre/post study).
Conducted by a native Thai author, the 45-60 minute semi-structured interviews used open-ended questions. 

\paragraph{Interview Coding}


We analyzed qualitative data using thematic analysis \cite{braun2006using}, collaboratively developing a codebook via inductive and deductive approaches \cite{saldana2021coding,kathleen2008team}.
Interviews, transcribed in Thai, were iteratively coded for recurring themes.
Two researchers independently coded a subset of transcripts for inter-coder reliability and resolved discrepancies through discussion.
Key themes included changes in driver awareness of air pollution, behavior changes during high PM2.5 periods, and participants' engagement in the project.
We then applied axial coding using the codebook to explore theme interrelationships \cite{williams2019art}.


