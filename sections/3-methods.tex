\section{Methodology}

% This section outlines the methods used in this study, including the deployment of real-time sensors for hyperlocal air pollution monitoring and a qualitative interview component to capture participants' lived experiences and perceptions.
% This study was approved by the Institutional Review Board (IRB) of University of Washington.
This study combined real-time sensor deployment for hyperlocal air pollution monitoring with qualitative interviews capturing participant experiences and perceptions.
University of Washington Institutional Review Board (IRB) approval was obtained.

\subsection{Participants Recruitment}

% Participants were recruited across both Bangkok and Chiang Mai to capture diversed driving behaviors and exposure patterns. 
% Criteria for recruitment including drivers had to be adults engaged in regular passenger drop-offs and drive at least five hours a day.
% Those who did not meet these criteria, such as individuals providing delivery services only, were excluded from the study.  
% In total, five drivers were recruited from each city, resulting in a sample size of ten participants. 
% \autoref{tab:demo} shows participants' income information.
Ten adult motorcycle rideshare drivers (five each from Bangkok/Chiang Mai) were recruited to capture diverse driving behaviors and exposure patterns.
Inclusion criteria were regular passenger transport and driving $\ge$5 hours/day; passengers-dropoff drivers were preferred.
Participant income details are in \autoref{tab:demo}.
% In Bangkok, recruitment initially used a ride-hailing app during rides; one legacy station-based taxi driver was added after dropouts.
% In Chiang Mai, due to fewer rideshare drivers, recruitment occurred via an informal \grab{} driver association.

\begin{table}[ht]
    \centering
    \caption[Participant Income]{Participant income sources (Rideshare: \grab{}; Legacy: station-based) and self-estimated daily average (THB).
    DriverID prefix indicates location (BKK: Bangkok, CMI: Chiang Mai).}
    \begin{tabular}{|c|l|c|}
        \hline
        \textbf{DriverID} & \textbf{Source of Income} & \textbf{Avg. Income/} \\
         &  & \textbf{Day (THB)} \\
        \hline
        BKK1  & Rideshare, Legacy & 2250 \\
        BKK2  & Legacy & 800 \\
        BKK3  & Rideshare, Legacy & 1300 \\
        BKK4  & Rideshare, Legacy & 1350 \\
        BKK5  & Rideshare & 1800 \\
        CMI1  & Rideshare & 700 \\
        CMI2  & Rideshare & 700 \\
        CMI3  & Rideshare & 600 \\
        CMI4  & Small Business Owner, Rideshare & 500 \\
        CMI5  & Rideshare & 900 \\
        \hline
    \end{tabular}
    % \caption{Reported sources of income (Rideshare for driving for \grab{} and Legacy for driving for legacy motorcycle taxi, where drivers wait for passengers at designated stations) and the self-estimated daily income in Baht.
    % % (approximately 34 Baht corresponds to 1 USD).
    % The DriverID contains an abbreviation for their location (BKK for Bangkok and CMI for Chiang Mai).}
    \label{tab:demo}
\end{table}

% \paragraph{Bangkok}
In Bangkok, participants were initially recruited through a widely used motorcycle ride-hailing application. 
Drivers were approached during rides, with an author (a native Thai speaker) explaining the study objectives and eligibility criteria. 
Due to last-minute participant dropouts, the research team recruited an additional motorcycle rideshare driver from the legacy rideshare system, where drivers wait for passengers at designated stations. 
% This hybrid approach ensured representation from both rideshare-dependent drivers and those operating outside of app-based systems.

% \paragraph{Chiang Mai}
In Chiang Mai, due to fewer rideshare drivers, recruitment occurred via an unofficial \grab{} driver association.
% Due to a sparse number of rideshare drivers and passengers in Chiang Mai, the recruitment was facilitated through an informal \grab{} driver association. 
% The association provided a direct communication channel to drivers who also drive for passengers.
% These drivers would have been otherwise harder to reach through formal recruitment methods, due to a sparse number of rideshare drivers and passengers who use the rideshare service. 
% This partnership allowed for the identification of participants who met the study criteria as well.

\paragraph{On-boarding Process}
% Drivers who expressed interest in participating were provided with a detailed explanation of the study, including its purpose, procedures, and data handling protocols.
% Participants were required to sign an informed consent form before being enrolled in the study.
% During the on-boarding process, the research team provided a walk-through on how to access and interpret their collected real-time data on the web application \todo{(provide citation for the web application )}
% The team explicitly clarified that the research team had no affiliation with any rideshare platform to mitigate concerns about potential conflicts of interest or data misuse. 
% We emphasize that all participants have the right to withdraw from the study or request deletion of their data at any time without penalty.

Interested drivers received study details (purpose, procedures, data handling) and provided informed consent.
Participants were required to sign an informed consent form before participating in the study.
Participants were trained on accessing/interpreting their real-time data via a map visualization~\cite{mapvis}.
We clarified non-affiliation with rideshare platforms and informed participants of their right to withdraw or request data deletion anytime without penalty.

% Participants were also informed that their data would be anonymized and used solely for academic purposes. 

% \reread{This balance allowed for meaningful comparisons across the two urban contexts while ensuring that data collection efforts were manageable, consistent, and fine-grain enough to provide insights at personal level.}

% Participants were recruited during rides. 
% In Bangkok, the authors requested via one the most-commonly used rideshare application.
% Due to the last minute dropout, we also one recruited motorcycle taxi driver without applicaiton (i.e., legacy taxi system where they wait for passenger at the corner of the street) 
% In Chiangmai, our partner recruited from the informal driver application association. 
% Exclusion criteria include drivers must be adults and do passenger drop-offs, those who did not meet the criterias are excluded.
% If the drivers expressed their interest, the drivers will sign consent. 
% Authors stressed during the recruitment that they have no association with the rideshare aplicaiton and only doing this for the research. 
% They were also ensured that they can decide to opt out from the study or request their data to be removed at anytime of the study. 


\paragraph{Compensation}
Participants were compensated 4,000 Baht per month for their participation, with payments distributed twice monthly during check-ins by transferring into their bank account.

\subsection{Real-Time Sensor Deployment and Calibration}

To collect real-time air pollution data, we deployed 10 portable low-cost air quality sensors developed by our research team.

% % To ensure measurement accuracy, each sensor underwent field testing by co-locating with a reference air quality station operated by the Pollution Control Department of Thailand (PCD) in Bangkok. 
% % We used this station to measure anc compare PM2.5 concentrations on hourly reference values. 
% % Over a 5-day period, data from three low-cost sensors demonstrated strong alignment with the PCD station, with Pearson correlation coefficients consistently above 0.81, indicating high accuracy and reliability in tracking PM2.5 variations. (citation will be provided after double-blind)

% \paragraph{Sensor Deployment}
% The sensors were mounted on motorcycle taxi drivers' helmets, positioned on top of the helmet near the breathing zone to capture roadside air conditions. Five drivers from Chiang Mai and five from Bangkok participated, each using a portable power bank to power the sensors throughout their day. 
% On average, drivers' working hours span from 6 to 17 hours daily; they turn on the sensors when leaving home and turn them off upon return (or until the powerbank runs out of battery). 
% Drivers also have access to a group chat with all the participants and research team on a widely-used messaging platform among Thai citizens, LINE.
% The notification system is also embedded in LINE.
% To preventing data loss, the notification system alerts each driver any malfunction
% or when their sensor becomes unresponsive, so they can conduct basic troubleshooting for contact the research team on time. 
% They can also see the real-time collected measurements from the web application, providing live updates of all participants. 

% The sensors were deployed from November 2023 to mid-May 2024, covering both peak and non-peak periods of Thailand's typical air pollution season (which is typically from January until April). 
% The deployment period also encompasses a full year’s cycle of the Thai seasons; rainy, winter, and summer. This temporal selection allows for the representation of diverse driving experiences across varying levels of PM2.5 exposure throughout the year. 

\paragraph{Sensor Deployment and Support}
Sensors were mounted on helmets of 10 motorcycle rideshare drivers (5 Chiang Mai, 5 Bangkok) near the breathing zone, powered by 2 portable batteries during their 6-17 hour workdays.
Drivers manually operated sensors and used a LINE group (a popular messaging app in Thailand) for communication and troubleshooting assistance, prompted by automated malfunction notifications.
They could view real-time collective data via a web application featuring a map visualization tool~\cite{mapvis} displaying color-coded pollution levels; training was provided.
% The LINE group also became a community space for sharing experiences [discussed in \autoref{sec:result}].
Deployment spanned November 2023 to mid-May 2024, covering Thailand's peak pollution season (Jan-Apr) and diverse seasonal conditions to capture varied PM2.5 exposure.

% \paragraph{Bi-Weekly Check-Ins and Calibration}
% Participants attended biweekly check-ins where they shared their experiences and received summary of the air quality data collected. 
% These sessions included a detailed overview of air pollution levels on the street in Chiang Mai and Bangkok. 
% The research team performed equipment checks, re-calibrated sensors when necessary, transferred data (i.e., the IMU data), and ensured that the sensors were functioning properly. 
\paragraph{Bi-Weekly Check-Ins}
Biweekly check-ins facilitated participant feedback, shared data summaries (local street-level pollution), equipment checks, and re-calibration.

% \paragraph{Hardware Support}
% All participants were invited to join a dedicated messaging group on the LINE platform, Thailand's most widely used messaging application. 
% % This group was initially created to provide 24/7 hardware support and address participant questions during the study. 
% % However, the group evolved into a dynamic community space where drivers shared experiences and insights about air pollution and their daily work, which we discussed in detail in \autoref{sec:result}.


% \paragraph{Map Visualization}
% The map visualization was one of the tools provided to the drivers, designed to help them access and interpret air quality data all the drivers collect (citation for the web application will be provided after the double-blind).
% The tool displayed real-time pollution levels using a color-coded system. Drivers could use the map to compare air quality across different cities, Bangkok and Chiang Mai, and track pollution trends over time. 
% The research team provided a walkthrough to ensure drivers understood how to navigate the interface, interpret the results, and use the data to make informed decisions about their routes and daily activities.

\subsection{Qualitative Interview}

% To complement the quantitative sensor data, qualitative exit interviews were conducted with all participants.
% The interviews aimed to explore the experiences of the drivers using sensor-equipped helmets, their driving routines, and their perceptions of air pollution before and after participating in the study.
% An author of native Thai language conducted all interviews to ensure cultural and linguistic alignment. Each session lasted 45-60 minutes and followed a semi-structured interviews with open-ended questions
% \firn{cite Appendix for details}.
Qualitative exit interviews with all participants explored their experiences, routines, and perceptions of air pollution (pre/post study).
Conducted by a native Thai author, the 45-60 minute semi-structured interviews used open-ended questions. 
% [cite Appendi]

\paragraph{Interview Coding}
% The qualitative data were analyzed using thematic analysis \cite{braun2006using}.
% The authors collaboratively developed a codebook based on a mix of inductive and deductive coding approaches \cite{saldana2021coding,kathleen2008team}.
% Initially, the interviews were transcribed in Thai, and an iterative coding process was conducted to identify recurring themes.
% To ensure inter-coder reliability, two independent researchers coded a subset of transcripts.
% Discrepancies were resolved through discussions. 

% The identified themes included drivers' perceived changes in air pollution awareness, behavior changes during high PM2.5 periods, and challenges using the sensor helmet. Using the codebook, the team conducted axial coding to explore relationships between themes \cite{williams2019art}.

We analyzed qualitative data using thematic analysis \cite{braun2006using}, collaboratively developing a codebook via inductive and deductive approaches \cite{saldana2021coding,kathleen2008team}.
Interviews, transcribed in Thai, were iteratively coded for recurring themes.
Two researchers independently coded a subset of transcripts for inter-coder reliability and resolved discrepancies through discussion.
Key themes included changes in driver awareness of air pollution, behavior changes during high PM2.5 periods, and challenges using the sensor helmet \todo{Do we discuss this?}.
We then applied axial coding using the codebook to explore theme interrelationships \cite{williams2019art}.


% This mixed methods approach provides a comprehensive understanding of the effects of air pollution on the lived experience of drivers, their behaviors, and their perceptions of health and safety.
