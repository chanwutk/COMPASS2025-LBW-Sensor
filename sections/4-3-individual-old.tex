% \subsection{Individual Insights -- Driver-Specific Motivations and Behaviors}
% \label{sec:result-individual}

% We incorporate data triangulation by integrating quantitative data, qualitative interviews, and group chat discussions to gain a comprehensive understanding of the strategies that drivers employ to mitigate their exposure to air pollution.

% We created composite timelines for each driver, visualizing pollution exposure alongside self-reported health and behavioral changes. 
We triangulated quantitative data, qualitative interviews, and group chat discussions to understand driver strategies for mitigating air pollution exposure.
Composite timelines \mick{I am not sure if 'composite timelines' is the write term to use here (assuming this means the comparison between the two drivers)} visualized each driver's exposure alongside self-reported health and behavioral changes.

\subsubsection{Driving Patterns}

Drivers attempted mitigation (e.g., BKK-4 used shortcuts to avoid congestion for both air quality and efficiency), but strategies were limited by passenger demand and algorithms.
% While some drivers adopted strategies to reduce exposure, these adjustments were often limited by passenger demand and algorithmic ride assignments. 
% BKK-4 particularly noted that he tried taking shortcuts and smaller alleys to actively attempted to avoid congestion, but not just due to avoid the poor air quality, but to also that we can make more rounds of pickups.
Ultimately,

\begin{quoteb}
    ``There is not much I can do. If the customer wants to go there, I'll have to go where they want to go.'' (BKK-4)
\end{quoteb}

% While \autoref{fig:hourly-work-aqi} the overall hourly average\kurtis{huh}, the plot showed minimal standard deviation during the day time.
% This suggests that despite individual efforts to navigate away from visibly polluted zones, drivers remained consistently exposed to harmful air pollution throughout their days.  
% While avoiding heavy traffic may prevent extreme localized exposure, the data highlights that poor air quality is pervasive, requiring broader systemic intervention. 
% This finding underscores the need for large-scale mitigation strategies beyond individual behavioral adjustments. 

% To better understand how driving behaviors influence exposure to air pollution, we analyzed two distinct approaches among drivers: the ``Selective'' or BKK-3, who focuses on avoiding high-pollution zones while still striving to maintain expected earnings, even if it means accepting slightly lower income. 
% The ``Income Maximizer'' ~\kurtis{?}or BKK-1, who primarily prioritizes maximizing earnings, aiming as much as 2,500 baht per day, often at the expense of exposure to polluted areas.  These contrasting behaviors illuminate how drivers navigate the complex challenge of balancing health risks with economic needs, revealing the different strategies they employ in their daily routines to cope with pervasive air quality issues.
Hourly exposure data (\autoref{fig:hourly-work-aqi}) showed minimal daytime variance,
indicating consistent exposure despite individual efforts.
While attempts to avoid traffic offer localized relief, pervasive pollution necessitates systemic interventions beyond individual actions.
We identified two archetypes navigating this health-economic trade-off: the ``Health-Conscious'' driver (e.g., BKK-3),
prioritizing some pollution avoidance over maximum income,
and the ``Income-Driven'' driver (e.g., BKK-1), prioritizing earnings (targeting ~2,500 baht/day) despite exposure.

\paragraph{Driver BKK-3: The ``Health-Conscious'' Driver} 
% % \todo{Add the graph comparing the exposure between the two drivers}

% Rain posed an immediate safety threat prioritized over pollution, often causing her to stop using the app entirely, consistent with prior findings on threat perception \cite{tieanklin2024rideshare}. While finding the real-time, localized AQI data useful (e.g., comparing Bangkok vs. Chiang Mai), she ultimately accepted pollution as unavoidable for income:

% \begin{quoteb}
%     ``We took this job for the money. We just have to live with it.'' (BKK-3)
% \end{quoteb}  

% BKK-3 illustrates the tension between financial need and health, showing limited individual control despite information tools. This highlights the need for structural solutions beyond individual adaptation.






% BKK-3, the 54-year-old female driver stood out for their heightened awareness of air pollution and the way she incorporated this knowledge into their driving decisions. 
% Unlike most of our participants who passively endured poor air quality, she actively sought ways to minimize their exposure—without fully abandoning the reality that pollution is an inherent part of their profession.

% During the interview, she revealed that she relied on television news for pollution updates. 
% However, after using the sensor-equipped helmet, their habits changed. She began frequently checking the web application we provided, tracking real-time AQI readings before and after their rides whenever they got a chance to. 
% This shift gave them a new understanding of how pollution levels fluctuate. 
% She noted how pollution spiked dramatically when riding behind big buses,  
BKK-3 (54, Female) actively incorporated heightened air pollution awareness into driving decisions, seeking exposure minimization strategies.
Initially relying on television news, she began frequently checking our web application for real-time AQI after using the sensor helmet, gaining insight into pollution fluctuations.
She observed pollution spikes when riding behind buses,

\begin{quoteb}
    ``When I follow big buses, the graph just shoots up immediately.'' (BKK-3)
\end{quoteb}  


% They also discovered that, even on seemingly clear days, AQI levels could reach hazardous levels of 80-100 µg/m$^3$. 
% This realization led her to make more selective choices when accepting ride requests. 
% While other drivers maximized their earnings by taking jobs in the busiest areas, she deliberately avoided them, opting instead for jobs that she could take a smaller alley - even if it meant traveling a slightly longer distance.  
and hazardous AQI levels (80-100 \textmu{}g/m$^3$) even on seemingly clear days.
This prompted her selective ride acceptance, e.g., avoiding busy areas for smaller alleys, despite potentially longer distances.
She stated,

\begin{quoteb}
    ``Main roads have more dusts (air pollutants), it’s better to take side streets.'' (BKK-3)
\end{quoteb}  


% They also adjusted their ride acceptance behavior. Near their home, they turned off the automatic ride-matching feature, manually selecting only those trips that would bring her closer to home, reducing prolonged exposure.  
and manually choosing trips toward home by disabling auto-match nearby to reduce prolonged exposure.

% Despite these adaptations, like with all of our participants, they acknowledged that avoiding air pollution entirely was unrealistic. On particularly bad days, she suffered from nasal irritation and a dry throat, sometimes deciding to return home earlier than planned, 
However, complete avoidance was impossible. Physical symptoms (nasal irritation, dry throat) sometimes led her to finish work early,

\begin{quoteb}
    ``I go home earlier quite often when I feel my nose stings and my throat feels dry.'' (BKK-3)
\end{quoteb}  


% BKK-3 further emphasized that rainy days brought an even greater sense of urgency. Unlike air pollution, which they could attempt to navigate, rain posed an immediate safety threat. 
% They frequently stopped using the app altogether during heavy rain, fearing accidents, which confirms prior research suggesting that visibility affects how people perceive threats \cite{tieanklin2024rideshare}.

% When asked whether air pollution monitoring had changed their perception of their work, she acknowledged that having real-time air quality updates was useful, particularly that she can see the most localized air quality measurement that directly impacted their and can compare pollution levels between Bangkok and Chiang Mai. 
% Yet, despite all her precautions, she ultimately accepted air pollution as an unavoidable aspect of the job, 
Rain posed an immediate safety threat prioritized over pollution, often causing her to stop using the app entirely, consistent with prior findings on threat perception \cite{tieanklin2024rideshare}.
While finding the real-time, localized AQI data useful (e.g., comparing Bangkok vs. Chiang Mai), she ultimately accepted pollution as unavoidable for income:

\begin{quoteb}
    ``We took this job for the money. We just have to live with it.'' (BKK-3)
\end{quoteb}  

% Their story illustrates the tension between financial necessity and personal well-being,
% highlighting how drivers may exercise some control over their exposure - but only to an extent. 
% While tools like real-time air quality map updates provide valuable insights, structural changes are needed to reduce exposure at a larger scale.  

BKK-3 illustrates the tension between financial need and health,
showing how drivers exercise some control over their exposure--but only to an extent. 
% showing limited individual control despite information tools.
% This highlights the need for structural solutions beyond individual adaptation.
While tools like real-time air quality map updates provide valuable insights, structural changes are needed to reduce exposure on a larger scale.  

% This driver employed a more cautious approach, actively avoiding high-risk routes identified through the monthly summaries provided by the research team. Consequently, Driver Y’s average PM2.5 exposure was [TODO: Insert Percentage]\% lower than the group average. However, this behavior led to reduced income, with Driver Y stating, ``I know the risks, but avoiding those areas means fewer rides and less money.''


\paragraph{Driver BKK-1: The ``Income-Driven'' Driver}
% This driver logged the highest daily kilometers, averaging [TODO: Insert Value] hours of driving per day. Real-time sensor data revealed prolonged exposure to PM2.5 levels exceeding 100~µg/m$^3$ for \todo{[TODO: Insert Percentage]}\% of the driving time. Despite awareness of the risks, Driver X reported, ``I need to keep driving to meet my daily income target, even if the air is bad.''

% On the other hand, BKK-1, the 46-year-old male motorcycle taxi driver who operates long hours from 4 AM to 9 PM daily, expecting earnings between 2,000 to 2,500 baht per day.  
% To supplement his income, he also takes on additional delivery jobs for regular customers.
% His primary focus is maximizing his daily earnings, which heavily influences his decisions and behaviors on the road.
% This financial drive overrides any considerations of air pollution exposure. As he explained, 
BKK-1 (46, Male), a motorcycle rideshare driver working long hours (4 AM-9 PM), targets daily earnings of 2,000-2,500 baht, supplementing with delivery jobs.
His primary motivation is maximizing income, which dictates his road behavior and overrides air pollution concerns,
as he explained, \mick{forgot to add quote?}

% Before participating in this study, BKK-1 admitted that he never followed news about air quality or paid attention to pollution levels. Since using the sensors, he has developed a habit of occasionally checking the air quality portal during his free time in the afternoons. 
% However, he primarily looks at the color-coded visualization rather than specific numerical values.

% Like many other drivers, BKK-1 reported to notice immediate short term impacts on his health due to air pollution, including symptoms like runny noses and eye irritation.
% % \firn{could add citations}.
% Despite these challenges, he affirmed that he always wears a mask while driving and believes this precaution is sufficient for protecting his health.
Initially ignoring air quality information, BKK-1 began occasionally checking visual air quality data from our map visualization after participating in the study.
He experiences short-term health effects like runny noses and eye irritation but believes wearing a mask provides sufficient protection.

\begin{quoteb}
    ``I can feel my body gets weaker (the more I drive), I always wear masks. ...If Google Maps tells us to go, I go. Wherever it is, I just have to push through.'' (BKK-1)
\end{quoteb}

% The awareness of highly polluted areas however does not influence his route choices. He emphasized that air pollution is pervasive across Bangkok and believes it is impractical to avoid specific areas based on pollution levels alone. Instead, his decisions are driven by income. Even when faced with irritation in his eyes and nose from poor air quality, he remains committed to following the app's directions to ensure he maximizes his earnings by reiterating that,
Despite increased awareness and symptoms, pollution data does not alter his route choices.
BKK-1 considers avoiding pollution impractical as pollution is pervasive across Bangkok and prioritizes income, strictly following app assignments regardless of location or pollution levels:

\begin{quoteb}
    ``I will not be able to be selective about routes; I go wherever the application assigns me to go, no matter of how far or how polluted the area may be.'' (BKK-1)
\end{quoteb}


% Although BKK-1 has made minor lifestyle changes since participating in this study - such as eating at home more often and increasing his use of protective gear.
% These adjustments are limited to personal habits and do not extend to his driving behavior. 
% He stated that nothing about the air pollution data has influenced his route choices or job selection as his focus is right on the income he makes that day primarily.  
While the study led to minor personal changes (e.g., eating at home more), his income-focused driving behavior remained unchanged, unaffected by air quality informed by our map visualization.



% % \firn{Figure compares the pollution exposure timelines for Driver X and Driver Y, illustrating how selective route adjustments impacted real-time exposure levels.}

% % \mick{todo: add a figure comparing the 2 drivers.}

% These two cases underscore the limitations of individual decision-making in the gig economy. 
% While access to air quality data may increase awareness, financial insecurity and platform constraints prevent drivers from prioritizing their health. 
% On the other hand, providing more financial incentives is unlikely to help these frontline workers to better mitigate the exposure to the prolonged air pollutions as
% % \todo{figure}
% shows no/low correlation between income made vs the pollution \todo{exposure...??}
% As this highlights, interventions must go beyond informational tools, addressing structural issues that limit a driver's capacity to exercise meaningful autonomy in their work.
% Government-led initiatives, such as expanding low-emission zones or integrating real-time pollution data into traffic management systems, could play a crucial role in reducing exposure for motorcycle taxi drivers.

These two cases highlight the limitations of individual decision-making in the gig economy. 
While access to air quality (AQ) data increases awareness, financial pressures and platform constraints prevent drivers from prioritizing health over income.
Financial incentives alone may be ineffective, given the weak correlation observed between earnings and pollution exposure.
\mick{we don't have this evidence}
Therefore, interventions must address structural barriers limiting driver autonomy, moving beyond information tools.
Systemic changes, such as low-emission zones or integrating real-time AQ data into traffic management, are crucial for reducing driver exposure.






% \subsubsection{Cross-Case Comparisons}
% \firn{to be decided if this should be kept}. Detailed case narratives highlight contrasting experiences:
% \begin{itemize}
%     \item \textbf{Driver X's Experience:} Despite being the most exposed, Driver X reported no immediate health complaints but described increased fatigue during peak pollution months.
%     \item \textbf{Driver Y's Experience:} Driver Y demonstrated heightened awareness of pollution risks and reported feeling less fatigue but expressed concerns about the financial trade-offs of avoiding polluted routes.
% \end{itemize}

\subsubsection{Increased Awareness, Public Engagement, and Empowerment}
% \mick{Can frame this differently. Instead of showing their pride, we should show how these participants help increase public awareness of PM.}

% \mick{can remove this (START)}
% Beyond their personal experiences with air pollution, many drivers expressed a profound sense of pride in participating in the study, particularly when asked about the sensors attached to their helmets. During bi-weekly check-ins, participants frequently shared moments when passengers, fellow drivers, or even pedestrians inquired about the device, especially during prolonged traffic congestion. 

% One of our Chiang Mai drivers (CMI-5) reflected on her experience using the air quality sensor, emphasizing that it did not make her driving more difficult. 
% She noted that the device was lightweight and, after an initial adjustment period, it became just another part of her routine.
% ``(The device) is small enough that I did not feel any air drag'' (CMI-5) she shared. 

% Beyond its function as a data-gathering device, the sensor became an unexpected conversation starter. 
% With a proud smile, CMI-5 described moments when pedestrians and fellow riders showed curiosity about the sensor. 


% \begin{quoteb}
%  ``People often stop and stare at it. Sometimes they ask if it's a WiFi router box!'' (CMI-5)
% \end{quoteb} 

% She described how pedestrians, intrigued by the small box on her helmet, would strike up conversations. 

% One particularly memorable encounter occurred at a traffic light, where a foreign motorcyclist gestured toward the sensor and looked very curious about the box on top of her helmet. 
% Struggling to find the words in English, she quickly turned to Google Translate. 

% \begin{quoteb}
%  ``I laughed and typed in the Google Translate to tell them that is an air quality sensor. He replied, \textit{`Very Good'}'' (CMI-5)
% \end{quoteb} 

% She shared the moment with a laugh and smile during the interview, explaining that the interaction not only gave her a sense of expertise and confidence in her knowledge about air pollution but also reinforced her awareness of the issue as people were often curious about the sensor.

% Furthermore, BKK-5 driver who have also heard about similar memorable encounter, proudly noted during the interview,

% \begin{quoteb}
%  ``Some of our group even got taken a picture of wearing the (sensor) helmet.'' (BKK-5)
% \end{quoteb}

% These spontaneous exchanges and unexpected encounters highlight how the presence of the sensor prompted broader awareness of air quality issues, not just among the drivers but also among the communities they navigated daily.

% \mick{can remove this (END)}

% They described instances where they warned colleagues about high pollution levels based on sensor readings, encouraging them to prepare protective measures.











% Most participants reported frequently checking the map visualization website, especially on days when air pollution levels were elevated. 
% Several drivers noted that their participation in the study had initiated meaningful discussions with friends and passengers about air quality.
% Such as CMI-4 mentioned advising fellow drivers to wear masks, while BKK-4 also shared real-time pollution readings with passengers, encouraging them to take protective measures when the real-time reading indicated high levels of pollution. 
% The visualization not only served as a personal reference for the drivers but also functioned as a tool for raising awareness within the broader community.
Most participants reported frequently checking the map visualization, particularly on high-pollution days.
Participation spurred air quality discussions with friends and passengers; for instance, CMI-4 advised mask use, while BKK-4 shared real-time data with passengers during poor conditions.
The visualization thus served both personal reference and community awareness.

% One driver, CMI-1, took the role to conduct information sharing especially seriously.
% During the interview, he proudly shared that he often took the initiative to inform other drivers in the province.
% He actively posted updates in the Chiang Mai \grab{} driver group, which includes more than 5,000 members. 

% He described his habit of monitoring air pollution levels in his mornings and sharing warnings with the group when conditions became hazardous. He specifically noted,  
Driver CMI-1 exemplified active information sharing, actively posting updates in a large Chiang Mai \grab{} driver group (>5,000 members).
He monitored morning pollution levels and alerted the group during hazardous conditions:

\begin{quoteb}
 ``I often take a screenshot when the air pollution levels turn red (hazardous) and share it in the group, warning everyone to be extra careful since the pollution got worsen that day.'' (CMI-1)
\end{quoteb}  


% This highlights how the real-time data from the sensor-equipped helmets extended beyond individual drivers.
% Instead of passively experiencing air pollution, some drivers, like CMI-1, became active contributors to a collective awareness effort, ensuring their community remained informed and vigilant. 
% These interactions reveal a growing awareness within the community, as drivers take on the role of informal educators, sharing vital information about air quality risks. 

% Having access to real-time data not only generates valuable environmental data, but also fosters awareness and informal knowledge sharing within their community.
% Drivers are not just passive recipients of data; they actively engage in discussions and raise situation awareness for people they encounter. 
This demonstrates how real-time data empowered drivers beyond personal monitoring, turning some, like CMI-1, into active contributors and informal educators fostering collective awareness.
Access to real-time data generated valuable environmental information and fostered community awareness through informal knowledge sharing.
Drivers became active participants, engaging in discussions and raising situational awareness for others, moving beyond passive data reception.




% All of our participants expressed enthusiasm about continuing their involvement in future research, viewing it as a meaningful contribution.

% \firn{review if we need this for the full paper}
% \begin{quoteb}
%     ``Thank you (the research team) so much. I've learned a lot for the past 7 months.'' (BKK-5)
% \end{quoteb}

% BKK-5 driver shared in the group chat, where BKK-4 also asked about the next phrase of the project and added,

% \begin{quoteb}
%     ``I'm looking forward to be a part of the team again.'' (BKK-4)
% \end{quoteb}



% Moreover, some drivers reported behavioral changes resulting from this awareness, such as increased use of protective measures or altered driving habits during high pollution days,

% \begin{quoteb}
% \todo{``Since I started using the sensor, I’ve noticed more drivers wearing masks during high pollution days.''}
% \end{quoteb}