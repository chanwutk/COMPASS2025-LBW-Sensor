\section{Conclusion}

This study investigated the severe PM2.5 exposure faced by motorcycle rideshare drivers in urban Thailand, 
highlighting how economic constraints create significant barriers for this population to follow standard health advice, even when aware of the dangers.
Building upon previously identified agency limitations, our seven-month mixed-methods study uniquely introduced a real-time air quality map visualization,
observing that although drivers incorporated this data to adjust routes and schedules,
income generation consistently superseded significant exposure mitigation actions.

Our contributions include fine-grained, individualized exposure data contextualized by lived experiences, revealing the complex interplay between health risks, economic pressures, and structural barriers.
The findings underscore the urgent need for multi-faceted interventions involving technology (e.g., real-time data interfaces), policy (e.g., financial support, targeted regulations), and platform adjustments.
