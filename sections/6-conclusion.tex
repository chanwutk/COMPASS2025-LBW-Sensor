\section{Conclusion}
% In this paper, we study motorcycle rideshare driver experiences regarding air pollution that the have to face daily,
% where we collect both quantitative data from 7-month long helmet-mounted air quality sensor and qualitative data from interviews with these drivers.

This study investigated the severe PM2.5 exposure faced by motorcycle rideshare drivers in urban Thailand, 
highlighting how economic constraints create significant barriers for this population to follow standard health advice, even when aware of the dangers.
% highlighting the inadequacy of standard health advice for this population due to economic constraints \joe{is this true? seems like they know PM2.5 is bad for them they just don't care. it's not really about a lack of advice, its an issue with poor people feeling forced to work despite negative health effects.}.
Building upon previously identified agency limitations, our seven-month mixed-methods study uniquely introduced a real-time air quality map visualization,
observing that although drivers incorporated this data to adjust routes and schedules,
income generation consistently superseded significant exposure mitigation actions.
% Through a seven-month mixed-methods approach integrating longitudinal personal sensing with qualitative interviews, we documented persistent exposure exceeding health guidelines and identified significant limitations on drivers' agency to mitigate these risks.
% \joe{change this statement to stand out more from previous paper. I'd argue these limitations have already been identified and published in CSCW 2024 >:)}

Our contributions include fine-grained, individualized exposure data contextualized by lived experiences, revealing the complex interplay between health risks, economic pressures, and structural barriers.
The findings underscore the urgent need for multi-faceted interventions involving technology (e.g., real-time data interfaces), policy (e.g., financial support, targeted regulations), and platform adjustments.
% Effectively protecting this vulnerable workforce requires a collaborative, multi-stakeholder approach focused on systemic changes that support both health and livelihood, moving beyond advice that individuals cannot realistically follow.