\section{Introduction}



Motorcycle rideshare drivers in Thailand, often low-income, face prolonged outdoor work in dense, polluted urban areas \cite{tieanklin2024rideshare}.
This vulnerability is amplified by high motorcycle usage (87\% of households in 2014 \cite{Poushter2015motorcyclestat}) and poor air quality \cite{iqr_rank}.
A key hazard is PM2.5, fine particulate matter linked to severe health risks, including respiratory and cardiovascular diseases and millions of premature deaths globally \cite{who_ambient_air_pollution}.
Prolonged exposure is associated with a 6-13\% increase in heart disease mortality per 10 $\mu g/m^3$ of PM2.5.

Standard public health advice for high PM2.5 (e.g., stay indoors and avoid busy roads) \cite{cdc_pm25} is often impractical for these workers.
Financial pressures and platform dynamics limit their agency to prioritize health over income, creating a gap between known risks and self-protection capabilities \cite{tieanklin2024rideshare}.




Understanding the specific, individual experiences of PM2.5 exposure and the barriers preventing workers from prioritizing their health is crucial.
This study aims to bridge this gap by focusing on the lived experiences of at-risk workers (including but not limited to rideshare drivers) in urban Thailand.
To better understand this intersection, we ask:
\begin{quote}
\textbf{How can integrating personal air pollution exposure data with the lived experiences of rideshare drivers help surface structural barriers, coping strategies, and health-income trade-offs and inform more equitable interventions that support worker agency and well-being?"}
\end{quote}

We employ a methodology combining longitudinal personal sensing with qualitative interviews to capture both the quantitative patterns and the qualitative nuances of exposure and agency over seven months.
By deploying personal sensors, we gather detailed, real-time data on individual PM2.5 exposure across varying conditions.
This quantitative data is complemented by in-depth qualitative interviews exploring personal narratives, health perceptions, and coping strategies.
We will present data illustrating that workers frequently experience exposure exceeding WHO guidelines \cite{who_aqg_2021}, highlighting the urgency of this issue.


    

Key contributions include:
\begin{enumerate}
    \item 
Fine-grained, longitudinal individual PM2.5 exposure monitoring.
    \item 
Integration of sensor data with qualitative insights into lived experiences and perceptions relative to health guidelines.
    \item 
Identifying challenges, coping strategies, and agency constraints through worker narratives.
    \item 
Exploring potential solutions considering health-income trade-offs. This research aims to inform interventions enhancing worker agency and well-being.
\end{enumerate}

























