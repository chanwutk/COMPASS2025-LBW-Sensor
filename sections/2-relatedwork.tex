\section{Related Work}
% In this section, we describe the past work exploring the exposure and effects of air pollution on various populations with different socio-economic status and its impacts on motorcycle rideshare workers.
We review prior work on air pollution exposure, socio-economic disparities, and impacts on motorcycle rideshare workers.
\todo{The related work section lacks depth; only four studies have been included in this section. It would be good if it included more studies that have studied a related phenomenon and synthesised how these studies inform the current study}
\mick{probably don't have time to find+add more related work}

\subsection{Rideshare Workers and Their Agency to Prioritize Health}
% Tieanklin et. al.~\cite{tieanklin2024rideshare} conducted a comprehensive mixed-methods study examining the social dynamics and autonomy of motorcycle taxi drivers in Thailand.
% The authors utilized both quantitative analysis of server-side rideshare logs and qualitative semi-structured interviews with drivers and passengers to explore how these drivers navigate their work environment, particularly in relation to air pollution exposure.
% The study revealed that despite the perceived autonomy offered by gig work, many drivers do not exercise this autonomy to avoid high-pollution events. 
% Financial pressures, social dynamics, and a lack of transparency regarding ride assignment algorithms significantly constrain their agency and decision-making. 
% The authors found that while drivers are aware of the health risks associated with prolonged exposure to air pollution, they often feel compelled to prioritize income over their well-being \cite{machado2021midlife,elfassy2019associations}. 
% This phenomenon is described as a paradox where drivers experience increased flexibility in their work hours but simultaneously lack the agency to make health-conscious decisions.

% The longitudinal nature of our study provides more in-depth insights into these dynamics over time, revealing patterns that static studies may overlook. By focusing on hyperlocal air pollution data, we contribute to a deeper understanding of the environmental challenges faced by rideshare drivers in urban settings, thereby enhancing the existing literature on gig economy workers and public health.

Tieanklin et al.~\cite{tieanklin2024rideshare} found that Thai motorcycle rideshare drivers often prioritize income over health despite pollution awareness.
While gig work offers flexibility, financial/social pressures and ride assignment opacity limit their agency to avoid pollution~\cite{machado2021midlife,elfassy2019associations}.
Our longitudinal study provides deeper temporal insights into these dynamics.
By incorporating hyperlocal data with real-time air quality monitoring sensors, we offer a granular understanding of the environmental challenges faced by these gig workers, enhancing public health perspectives.


% \todo{adding another related work on technology/or data needed for ridershare drivers to make decision}



\subsection{Air pollution in Bangkok, Thailand}
% Nguyen et. al.~\cite{nguyen2023bangkokpollution} investigate the relationship between air pollution exposure, socio-economic status, and working and living conditions in Bangkok.
% By combining air-quality monitoring and a survey of 400 participants across five districts, it identifies disparities in exposure levels and public perceptions of air pollution.
% The findings reveal that informal workers and residents of peri-urban districts with poor ventilation experience higher exposure and more severe health impacts.
% Public awareness about local air pollution is low, and perceptions often diverge from official monitoring data.
% The study concludes with six policy recommendations, including better public communication, healthcare, and air pollution control measures.
Nguyen et al.~\cite{nguyen2023bangkokpollution} linked socio-economic status (SES) to air pollution exposure and health impacts in Bangkok, finding that informal workers were highly affected and public awareness was low or misaligned with official data.
% \cite{nguyen2023bangkokpollution} highlights the effects of air pollution regarding socio-economic inequality and points out public perceptions of air pollution at large.
% These findings inspired us to further explore these effects on sub-populations with low socio-economic status and unavoidable air pollution exposure, specifically, motorcycle rideshare drivers as one of the sub-populations that are affected the most by this air pollution crisis.
% To represent this sub-population's daily air pollution exposure,
% our study measured the air quality surrounding individuals within the sub-population.
% To further deepen the insight from the sub-population's personal experiences, we explore the decision-making process as a result of their perceptions of air pollution through participants' interviews.
Building on the findings on socio-economic inequality and public perceptions regarding air pollution, we focus on motorcycle drivers—a highly exposed, low-SES group.
Our study measures their hyperlocal air pollution exposure and uses interviews to explore how their perceptions influence decision-making.

% \subsection{Poverty resulting in Air Pollution}
% The research conducted by Rentschler and Leonova investigates the complex relationship between air pollution exposure and poverty across 211 countries and territories \cite{rentschler2022air}. By employing high-resolution PM2.5 data alongside subnational poverty estimates, the authors reveal that over 7.3 billion individuals live in regions where pollution levels exceed the World Health Organization (WHO) recommended thresholds, with approximately 80\% of these individuals residing in low- and middle-income countries.
% This study highlights significant disparities in exposure to air pollution, particularly among low-income populations who are disproportionately affected due to their reliance on outdoor labor and limited access to healthcare services. The findings provide critical insights into the global burden of air pollution, emphasizing how socio-economic inequality exacerbates vulnerability to health risks.
% However, our research offers a more nuanced understanding of these dynamics through a longitudinal data collection approach over a seven-month period, focusing specifically on hyperlocal air pollution levels in the streets of Bangkok and Chiang Mai. This methodology allows us to capture temporal variations and localized exposure patterns that broader studies may overlook.
% In contrast to Rentschler and Leonova's reliance on aggregated subnational poverty estimates, our study provides granular data that can reveal intra-urban disparities, such as those affecting motorcycle taxi drivers who are frequently exposed to high pollution levels due to their occupational environments. Furthermore, our research examines how air pollution exposure interacts with various occupational risk factors beyond income, providing a comprehensive view of the health risks faced by urban transportation workers.
% By highlighting these aspects, our study not only enhances the existing literature on air pollution and socio-economic inequality but also informs more effective policy interventions tailored to address the specific needs of vulnerable populations in urban settings.
