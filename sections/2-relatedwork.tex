\section{Related Work}
We review prior work on air pollution exposure, socio-economic disparities, and impacts on motorcycle rideshare workers.


\subsection{Rideshare Workers and Their Agency to Prioritize Health}


Tieanklin et al.~\cite{tieanklin2024rideshare} found that Thai motorcycle rideshare drivers often prioritize income over health despite pollution awareness.
While gig work offers flexibility, financial/social pressures and ride assignment opacity limit their agency to avoid pollution~\cite{machado2021midlife,elfassy2019associations}.
Our longitudinal study provides deeper temporal insights into these dynamics.
By incorporating hyperlocal data with real-time air quality monitoring sensors, we offer a granular understanding of the environmental challenges faced by these gig workers, enhancing public health perspectives.





\subsection{Air pollution in Bangkok, Thailand}
Nguyen et al.~\cite{nguyen2023bangkokpollution} linked socio-economic status (SES) to air pollution exposure and health impacts in Bangkok, finding that informal workers were highly affected and public awareness was low or misaligned with official data.
Building on the findings on socio-economic inequality and public perceptions regarding air pollution, we focus on motorcycle drivers—a highly exposed, low-SES group.
Our study measures their hyperlocal air pollution exposure and uses interviews to explore how their perceptions influence decision-making.

