\section{Related Work}
We review prior work on air pollution exposure, socio-economic disparities, and impacts on motorcycle rideshare workers.

\subsection{Rideshare Workers and Their Agency to Prioritize Health}



Motorcycle taxi drivers face high exposure to air pollution daily but are often excluded from formal occupational safety regulations due to self-employment or contract status~\cite{slater2022air}. 
Tieanklin et al.~\cite{tieanklin2024rideshare} found that Thai motorcycle rideshare drivers often prioritize income over health despite pollution awareness.
While gig work offers flexibility, financial/social pressures and ride assignment opacity limit their agency to avoid pollution~\cite{machado2021midlife,elfassy2019associations}.
Therefore, understanding their activity patterns is key to identifying their mitigation strategies. 
By incorporating hyperlocal data with real-time air quality monitoring sensors and their locations, we offer a granular understanding of the environmental challenges faced by these gig workers, enhancing public health perspectives.





\subsection{Understanding Risk Through Personal Data}

Environmental health risks are unevenly distributed, with informal and low-income workers often facing higher exposure and fewer protections. 
While personal informatics systems have been developed to support health and self-awareness through data tracking~\cite{li2010stage}, these systems largely focus on white-collar users with stable routines~\cite{epstein2020}. 
For informal workers like rideshare drivers - whose labor is mobile, physically demanding, and shaped by economic precarity - the role of personal data in shaping awareness and behavior remains underexplored~\cite{slater2022air}.

Nguyen et al.~\cite{nguyen2023bangkokpollution} show that low-SES workers in Bangkok are disproportionately affected by air pollution and often lack access to accurate or usable information. 
This limits their ability to understand their own exposure or to take informed, protective action. 
In this study, we seek to address that gap by supporting drivers in making sense of their own exposure data - helping them connect them with their personalized working conditions, and revealing how they cope with risk under constrained conditions. 
By integrating hyperlocal sensor data with interviews, we surface not only exposure patterns but also the decisions, adaptations, and structural limitations that shape how drivers respond. 
This work contributes to understanding how environmental health risk is experienced and navigated in the context of informal labor.








