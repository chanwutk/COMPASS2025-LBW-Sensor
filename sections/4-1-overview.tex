\section{Overview of Collected Data}
\label{sec:result-overview-data}

\paragraph{Air Quality Data From Sensors}
Over the seven-month period (November 2023 to mid-May 2024), we collected data from ten motorcycle rideshare drivers in Bangkok and Chiang Mai, covering peak and non-peak pollution seasons.
Each driver wore a helmet-mounted low-cost PM2.5 sensor, calibrated against stationary PCD air quality towers in Chiang Mai and Bangkok.

In total, we captured over 50 million data points on street-level pollution in approximately 163.49 km$^2$ driven in Bangkok and 97.21 km$^2$ in downtown Chiang Mai.
Drivers drive approximately 6-10 hours per day, with consistent sensor functionality powered by portable power banks.

\paragraph{Sensor Data Cleaning and Validating}
Prior the deployment, each PM2.5 sensor was field-tested by co-locating it with a reference air quality station operated by Thailand’s Pollution Control Department (PCD) in Bangkok.
Over five days, three low-cost sensors showed strong alignment with hourly PM2.5 readings from the PCD station, with Pearson correlation coefficients consistently above 0.81, indicating high reliability. 

After collecting the data, we removed the PM2.5 readings above the detection limit (500 $\mu g/m^3$). 
Some drivers also received replacement sensors due to hardware issues; we reconciled device changes by mapping data streams to the correct participant timelines. 



\paragraph{Exit Interviews}
We conducted qualitative exit interviews with all participants to supplement the sensor data. 
These interviews focused on understanding drivers' experiences with sensor-equipped helmets, their daily driving patterns, and any changes in awareness or behavior regarding air pollution. 
The author, a native Thai speaker, conducted the 60-90 minute interviews, using a semi-structured approach with open-ended questions. 
