\section{Overview of Collected Data}
\label{sec:result-overview-data}

\paragraph{Air Quality Data From Sensors}
Over the seven-month period (November 2023 to mid-May 2024), we collected data from ten motorcycle rideshare drivers in Bangkok and Chiang Mai, covering peak and non-peak pollution seasons.
Each driver wore a helmet-mounted low-cost PM2.5 sensor, calibrated against stationary PCD air quality towers in Chiang Mai and Bangkok.
% \kurtis{picture?}
% Data is uploaded in a GitHub repository \todo{add ref}.
In total, we captured over 50 million data points on street-level pollution in approximately 163.49 km$^2$ driven in Bangkok and 97.21 km$^2$ in downtown Chiang Mai.
Drivers drive approximately 6-10 hours per day, with consistent sensor functionality powered by portable power banks.

% \mick{should also include the interview results as data collected too.-added}
% \mick{we have XX interview records-added}
% \mick{we have enter/exit interviews?-added}
% \mick{we have rider's LINE group activity data (with their consent for sharing?)-added}

% \paragraph{Group Communication}
% % \todo{no need to repeat that the group was initially for technical support.}
% To facilitate ongoing communication, the team created a group chat with all of the 10 participants and the research team
% for 24/7 support for the participants, regarding the study.
% % While the group was initially established as a channel for participants to ask questions about the study or seek 24/7 support for troubleshooting hardware issues, it quickly evolved into an essential community space.  
% Participants
% % from both Bangkok and Chiang Mai
% began
% % using the chat not only to share their experiences but also
% % to discuss unexpected challenges, such as technical issues or route changes, and, most notably,
% to share and reflect on the air pollution levels they encountered during their rides.
% % This interaction fostered a sense of camaraderie among participants and creating a shared repository of lived experiences related to air pollution exposure. 
% The organic development of this virtual community provided a valuable layer of qualitative data that supplemented our interview findings and highlighted the collective concerns and strategies adopted by the drivers.

\paragraph{Exit Interviews}
We conducted qualitative exit interviews with all participants to supplement the sensor data. 
These interviews focused on understanding drivers' experiences with sensor-equipped helmets, their daily driving patterns, and any changes in awareness or behavior regarding air pollution. 
The author, a native Thai speaker, conducted the 60-90 minute interviews, using a semi-structured approach with open-ended questions. 